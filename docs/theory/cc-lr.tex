\documentclass{article}
\usepackage{amsmath, amssymb, physics}
\usepackage[backend=biber, style=phys]{biblatex}
\addbibresource{/home/pawel/Documents/Zotero/pawels_library.bib}
\usepackage{hyperref}
\hypersetup{
    colorlinks=true,
    linkcolor=cyan,
    citecolor=magenta,
    filecolor=magenta,
    urlcolor=cyan,
    runcolor=cyan
}
\usepackage{doi}

\title{Coupled Clusters Response Properties}
\author{Pawe\l{} W{\'o}jcik}

\begin{document}
\maketitle

\section{CC equations}
The CC energy is given by 
\begin{equation}
    E _{CC} = 
    \bra{\text{HF}}
    e ^{-T}
    H 
    e ^T
    \ket{\text{HF}}.
    \label{eq:cc_energy}
\end{equation}
The CC equations (in the linked
formulation\autocite{helgakerMolecularElectronicstructureTheory2000}) are given
by
\begin{equation}
    \bra{\mu} e^{-T} H e^T \ket{0} = 0
    \label{eq:cc_equations}
\end{equation}
The CC vector is given by
\begin{equation}
    \ket{\text{CC}} = e ^T \ket{\text{HF}}.
    \label{eq:cc_vector}
\end{equation}
The cluster operator for an $N$-electron system is given by
\begin{equation}
    T = T _1 + T _2 + \ldots + T _N,
    \label{eq:T}
\end{equation}
\begin{equation}
    T _1 
    = 
    \sum _{ai} \sum _ {\sigma_1 \sigma _2}
    t _{a\sigma _1 i \sigma _2}
    a ^\dagger _{a \sigma _1} a _{i \sigma _2},
    \label{eq:T1}
\end{equation}
\begin{equation}
    T _2 
    = 
    % \frac{1}{(2!)^2}
    \sum _{abji} \sum _ {\sigma_1 \sigma _2 \sigma_3 \sigma _4}
    t _{a\sigma _1 b \sigma _2 j \sigma_3 i \sigma _4}
    a ^\dagger _{a \sigma _1} 
    a ^\dagger _{b \sigma _2} 
    a _{j \sigma _3}
    a _{i \sigma _4}.
    \label{eq:T2}
\end{equation}
The chemistry convention specifies that the indices $i$, $j$, $k$, $l$ index
the occupied orbitals, while the indices $a$, $b$, $c$, $d$ index the virtual
orbitals. Occupied orbitals are the ones that form the reference state
$\ket{\text{HF}}$ while virutal orbitals are the reminder. In a shorthand
notation, the products of creation and annihilations operators are replaced
with a corresponding excitation operator $\tau$
\begin{equation}
    T = t \tau = \sum _\nu t _\nu \tau _\nu.
    \label{eq:T_tau}
\end{equation}
The index of the excitation operator goes over single excitations, double
excitations, and so on.

Koch and J{\o}rgensen\autocite{kochCoupledClusterResponse1990} also define the
excited bras
\begin{equation}
    \bra{\nu} = \bra{\text{HF}} \tau ^\dagger _\nu,
    \label{eq:bra_nu}
\end{equation}
and the excited bras with an extra exponential
\begin{equation}
    \bra{\bar{\nu}} 
    = \bra{\nu} e ^{-T}
    = \bra{\text{HF}} \tau ^\dagger _\nu e ^{-T}.
    \label{eq:bra_nu_bar}
\end{equation}

\section{Lambda equations}
The dual of the CC vector, $\bra{\text{CC}}$, is no good for calculation of
properties, because it breaks the Hellman-Feynman
theorem.\autocite{helgakerMolecularElectronicstructureTheory2000} To fix this
issue, the CC theory founders have created an alternative dual vector to
$\ket{\text{CC}}$, which assures the Hellman-Feynman theorem is satisfied and
the properties make sense. The dual vector is typically called lambda,
$\bra{\Lambda}$.

The lambda is defined as
\begin{equation}
        \bra{\Lambda}
        =
        \bra{\text{HF}} 
        +
        \sum _{\nu}
        \zeta _\nu \bra{\nu} e^{-T}
    =
    \left(
        \bra{\text{HF}}
        +
        \sum _{\nu}
        \zeta _\nu \bra{\nu} 
    \right)e^{-T}
    \label{eq:lambda}
\end{equation}
The $\zeta _\nu$ coefficients are determined by making sure that the
$\bra{\Lambda}$ satisfies the Schr\"odinger equation
\begin{equation}
    \bra{\Lambda} H = \bra{\Lambda} E _{CC}.
    \label{eq:lambda_se}
\end{equation}
To find the equations for $\zeta _\nu$, first the equation
Eq.~\eqref{eq:lambda_se} is hit with the cluster operator from the right
\begin{equation}
    \bra{\Lambda} H e^T = \bra{\Lambda} e^T E _{CC},
    \label{eq:lambda_se_eT}
\end{equation}
then it is projected onto the reference determinant, $\ket{\text{HF}}$, and
enough many excited determinants, $\ket{\mu}$. Projection onto
$\ket{\text{HF}}$ resuts in an equation that vanishies, for the $E _{CC}$ and
$T$ satisfying the CC equations. Projections onto the excited determinants
lead to
\begin{equation}
    \bra{\Lambda}H e^T \ket{\mu}
    =
    E _{CC}
    \bra{\Lambda}
    e ^T
    \ket{\mu}.
    \label{eq:zeta_eq}
\end{equation}
The rhs simplifies to $E _{CC} \zeta _\mu$, leading to the lambda residual
equations
\begin{equation}
    \bra{\Lambda}H e^T \ket{\mu}
    -
    E _{CC} \zeta _{\mu}
    = 
    0.
    \label{eq:lambda_residual}
\end{equation}


\section{Linear Response Properties}
The linear response properties describe the response of a molecule to an
external field. This statement is formalized by writing down the system
Hamiltonian as
\begin{equation}
    H_{sys} = H + V
    \label{eq:H_plus_pert}
\end{equation}
I follow the method of Koch and
J{\o}rgensen.\autocite{kochCoupledClusterResponse1990} Much of the notation from
this work is explained in detail in an earlier work of Olsen and
J{\o}rgensen\autocite{olsenLinearNonlinearResponse1985}, as well as in the book
by Zubarev.\autocite{zubarevNonequilibriumStatisticalThermodynamics1974}

The coupled cluster linear response function is given by Eq.~(94) from
Ref.~\cite{kochCoupledClusterResponse1990}
\begin{equation}
    \expval{\expval{A; B}} _{\omega _1}
    =
    \sum _\mu
        \bra{\Lambda}
            \commutator{A}{\tau _\mu}
        \ket{CC}
        X ^B _\mu (\omega _1)
    +
    \sum _\mu
    \left(
        \bra{\Lambda}
            \commutator{B}{\tau _\mu}
        \ket{CC}
        +
        \sum _\gamma
            F _{\mu \gamma}
            X ^B _\gamma (\omega _1)
    \right)
    X ^A _\mu (-\omega _1)
    \label{eq:lin_res}
\end{equation}
The matrices $X ^x _\mu (\omega)$ are the $t$ amplitudes responses defined in
Eq.~(95) and (58) or Ref.~\cite{kochCoupledClusterResponse1990}.
\begin{equation}
    X ^x _\mu (\omega)
    = 
    \sum _\nu
    (- \mathbf{A} + \omega \mathbf{I}) ^{-1} _{\mu \nu}
    x _\nu,
    \label{eq:t_response}
\end{equation}
where the $x _\nu$ matrix is related to the external perturbation operator,
$x$ (e.g., for the electric dipole perturbation this operator is the electric
dipole operator $x = \hat{\mu}$, see
Ref.~\cite{olsenLinearNonlinearResponse1985})
\begin{equation}
    x _\nu = \bra{\nu}e ^{-T} x e^T \ket{\text{HF}},
    \label{eq:cc_mu}
\end{equation}
and the matrix $\textbf{A}$ is the CC Jacobian
\begin{equation}
    A _{\mu \nu} 
    =
    \bra{\mu}
    e^{-T}
    \commutator{H}{\tau _\nu}
    e ^T
    \ket{\text{HF}}
    \label{eq:cc_jacobian}
\end{equation}
Finally, the $F _{\mu\nu}$ matrix originates from the $\zeta$ amplitudes
response and is given by Eq.~(77) from
Ref.~\cite{kochCoupledClusterResponse1990}
\begin{equation}
    F _{\mu \gamma}
    =
    \bra{\Lambda}
    \commutator{\commutator{H}{\tau _\mu}}{\tau _\gamma}
    \ket{\text{CC}}.
    \label{eq:fmatrix}
\end{equation}

One more term from the Eq.~\eqref{eq:lin_res} has a shortcut notation
(following Eq.~(76) from Ref.~\cite{kochCoupledClusterResponse1990})
\begin{equation}
    \eta ^x _\nu (\omega) = 
    \bra{\Lambda}
    \commutator{x ^\omega}{\tau _\nu}
    \ket{\text{CC}},
    \label{eq:eta}
\end{equation}
where $x ^\omega$ is a Fourier transform component of the external field
operator. Using this shortcut the Eq.~\eqref{eq:lin_res} becomes
\begin{equation}
    \expval{\expval{A; B}} _{\omega _1}
    =
    \sum _\mu
        \eta ^A _\mu
        X ^B _\mu (\omega _1)
    +
    \sum _\mu
    \left(
        \eta ^B _\mu
        +
        \sum _\gamma
            F _{\mu \gamma}
            X ^B _\gamma (\omega _1)
    \right)
    X ^A _\mu (-\omega _1).
    \label{eq:lin_res2}
\end{equation}

\printbibliography{}
\end{document}
